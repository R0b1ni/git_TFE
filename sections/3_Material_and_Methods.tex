\section{Blade design}
The design of turbine blades is a critical aspect that directly influences the performance and efficiency of the turbine. We will here use the method proposed by ...(Abeykoon) for the blade design of a Kaplan turbine.


The first step is to estimate the extracted power of the turbine to start the design.
Based on the input data, we can start by estimating the blockage factor with the following graph(from fig 15 of 2022 Abeykoon):
\begin{itemize}
    \item $\sigma = 1.6$
\end{itemize}
Considering: (from fig 15 of 2022 Abeykoon)
\begin{equation}
    R_i = 0.4 \times R_o = 0.04 \, \mathrm{m}
\end{equation}
We also obatin that we have 4 blades for the turbine, which is a common choice for Kaplan turbines.

We can calculate the volumetric flowrate $Q$:
\begin{equation}
    Q = \pi (R_o^2 - R_i^2) V_x
      = \Qvol \, \mathrm{m^3/h}
\end{equation}

By taking the maximal theorical hydraulic efficiency for a Kaplan turbine,
we can estimate the extracted power:
\begin{equation}
    P_{extracted} = \eta P_{hydraulic} = \eta \rho g Q \Delta H
\end{equation}
\begin{equation}
    P_{extracted} = \Pext \, \mathrm{W}
\end{equation}


First, we can calculate the rotationnal speed based on the following correlation: (Abeykoon)
\begin{equation}
N = \sigma \frac{(2gH)^{3/4}}{2\sqrt{\pi Q}}
\end{equation}
Therefore, we have:
\begin{equation}
N = \N \, \mathrm{rpm}
\end{equation}
We can also obtain the mean speed of the blade:
\begin{equation}
U_m = 2\pi R_m \frac{N}{60}
\end{equation}
with
\begin{equation}
R_m = \frac{(R_o - R_i)}{2} = 0.07 \, \mathrm{m}
\end{equation}

\subsection{1D Meanline Design}
With all these parameters, we can start the 1D meanline blade design. The first step is to set the angles, and different types of speed : \\
To insert : velocity triangles, angles, etc...

From the mass conservation theorem, we know that the axial velocity at the inlet of the turbine is equal to the axial velocity at the outlet of the turbine, therefore we can write:
\begin{equation}
c_{2x,m} = c_{1x,m} = c_{x,pipe} = 2 \, \mathrm{m/s}
\end{equation}

Then, we will calculate $c_{2 \theta,m}$ velocity at the outlet of the turbine. To do this, we will use the Euler formula:
\begin{equation}
P = \omega T = \frac{U_m}{R_m} T 
\end{equation}
From the second law of motion, we can write for a constant radius:
\begin{equation}
T = \dot{m} R_m (c_{2 \theta,m} - c_{1 \theta,m}) 
\end{equation}
which we can rewrite:
\begin{equation}
P = \dot{m} U_m (c_{2 \theta,m} - c_{1 \theta,m}) 
\end{equation}
and we finally find $c_{2 \theta,m}$ knowing that $c_{1 \theta,m} = 0$:
\begin{equation}
c_{2\theta,m} = \frac{P}{\dot{m} U_m}
\end{equation}

We can now calculate the relatives velocities knowing that:
\begin{equation}
\vec{w} = \vec{c} + \vec{U}
\end{equation}
\begin{equation}
\vec{c} = \vec{c_x} + \vec{c_\theta}
\end{equation}
We then have:
\begin{equation}
w_{1,m} = \sqrt{c_x^2 + U^2}
\end{equation}
\begin{equation}
w_{2,m} = \sqrt{c_x^2 + (U + c_{2\theta,m})^2}
\end{equation}


We can therefore plot the velocity triangles at the inlet and outlet of the turbine, for the mean radius: 
\begin{figure}[H]
    \centering
    \includegraphics[width=0.8\textwidth]{Part1/results/Velocity_Triangles.pdf}
    \caption{Velocity triangles at the inlet and outlet of the turbine, for the mean radius}
    \label{fig:velocity_triangles}
\end{figure}

\subsection{3-D design}
Once we have the velocity triangles at $R_m$, the next step is to divide the blade into several sections and calculate the velocity triangles at each section as we can see on the following figure:

To insert : blade sections 

We can choose a vortex distribution for the blade design.
Here, we chose a free vortex distribution, which means that the tangential velocity at the inlet of the turbine will be inversely proportional to the radius and we obtain a constant work along the blade. We can therefore write:
\begin{equation}
c_{2 \theta}(r) = \frac{K}{r}
\end{equation}
with $K$ a constant that we can calculate with the velocity triangle at the mean radius:
\begin{equation}
K = c_{2 \theta,m} R_m
\end{equation}
It's important to notice that with this choice of vortex distribution, we will obtain a twisted blade as the velocity triangles will be different at each radius. Knowing $c_{2 \theta}(r)$ allows us to calculate the velocity triangles at each radius, and therefore the angles of the blade at each radius by the same methodology as before.\\
For each radius $r_i$, the blade speed can be calculated as:
\begin{equation}
U_i = 2 \pi r_i \frac{N}{60}
\end{equation}
The velocity triangle is then calculated as:
\begin{equation}
\left\{
\begin{aligned}
c_{2x,i} &= c_{1x,i} = c_{x} = 2 \, \mathrm{m/s} \\
c_{2\theta, i} &= c_{2\theta}(r_i) = \frac{K}{r_i} \\
c_{1,i} &= \sqrt{c_{x}^2 + c_{1\theta,i}^2} = c_x \\
c_{2,i} &= \sqrt{c_{x}^2 + c_{2\theta,i}^2} \\
w_{1,i} &= \sqrt{c_{x}^2 + U_i^2} \\
w_{2,i} &= \sqrt{c_{x}^2 + \left(U_i + c_{2\theta, i}\right)^2}
w_{\inf , i} = (w_{1,i} + w_{2,i})/2
\end{aligned}
\right.
\end{equation}
We can also calculate the angles of the blade at each radius using basic trigonometry. We end up with the following final values:\\
To insert : table of angles, velocities, etc... at each radius

Then, we will need to define the chord to pitch ratio for each radius, which will allow us to calculate the chord length at each radius. We will use the values from (Abeykoon) for a fast runner, therefore we have $\frac{s^{\prime \prime}}{t^{\prime \prime}}$ varying between $0.75 -1.3$ depending on the radius. \\
Since we have 4 blades, we can finally calculate the pitch as:
\begin{equation}
t_i^{\prime \prime} = \frac{2 \pi r_i}{4}
\end{equation}
The chord is then simply calculated as:
\begin{equation}
    s_i^{\prime \prime} = t_i^{\prime \prime} \times \left[ {\frac{s^{\prime \prime}}{t^{\prime \prime}}}\right]_i
\end{equation}

To insert : table of chord and pitch at each radius

\subsection{Blade Geometry}

With all the values calculated in the previous sections, we can now start defining the geometry of the blade. We will use a methodology based on Janjua et al. (2013)